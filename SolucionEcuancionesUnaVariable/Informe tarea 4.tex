\documentclass[12pt]{article}

\title{Informe Tarea 4}

\author {Ronald Cardona
\and Anderson Grajales
\and Sebastian Valencia
\and Julian Sanchez}

% Packages
\usepackage[spanish]{babel}
\selectlanguage{spanish} 
\usepackage[spanish, onelanguage]{algorithm2e} %for psuedo code
\usepackage[utf8]{inputenc}


\begin{document}

\maketitle


\section{Solución numérica de ecuaciones de una variable}

En el presente documento se describen algunos de los algoritmos numéricos principales para resolver ecuaciones de una variable. Estos algoritmos nos permiten ya sea encontrar intervalos en los que existe una raíz de la ecuación $f(x) = 0$ ó encontrar propiamente la raíz.

\subsection{Búsqueda Incremental}

\begin{algorithm}[H]
	\caption{Algoritmo de Búsqueda Incremental}
	\SetAlgoLined
	Leer $x_0$, $delta$, $niter$ \\
	$fx0$ $\leftarrow$ $f(x_0)$ \\
	\eIf{$fx0 = 0$}
	{$x_0$ es raíz}
	{
		$x_1 \leftarrow x_0 + delta$ \\
		$contador \leftarrow 1$ \\
		$fx1 \leftarrow f(x_1)$ \\
		\While{$fx0 * fx1 > 0$ \textbf{y} $contador < niter$ }
		{
			$x_0 \leftarrow x_1$ \\
			$fx0 \leftarrow fx1$ \\
			$x_1 \leftarrow x_0 + delta$ \\
			$fx1 \leftarrow f(x_1)$ \\
			$contador \leftarrow contador + 1$
		}
		\uIf{$fx1 = 0$}
		{$x_1$ es raíz}
		\uElseIf{$fx0 * fx1 < 0$}
		{Hay raíz entre $x_0$ y $x_1$}
		\Else{Fracasó en $niter$ iteraciones}
	}
\end{algorithm}

\subsection{Método de la bisección}

\begin{algorithm}[H]
	
	\caption{Método de la Bisección}
	\SetAlgoLined
	Leer $x_i$, $x_s$, $tolerancia$, $niter$ \\
	$fxi \leftarrow f(x_i)$ \\
	$fxs \leftarrow f(x_s)$ \\
	\uIf{$fxi = 0$}
	{$x_i$ es raíz}
	\uElseIf{$fxs = 0$}
	{$x_s$ es raíz}
	\uElseIf{$fxi * fxs < 0$}
	{
		$x_m \leftarrow \frac{x_i + x_s}{2}$\\
		$fxm = f(x_m)$\\
		$contador \leftarrow 1$\\
		$error \leftarrow tolerancia + 1$ \\
		\While{$error > tolerancia$ \textbf{y} $fxm \neq 0$ \textbf{y} $contador < niter$}
		{
			\eIf{$fxi * fxm < 0$}
			{$x_s \leftarrow x_m$\\
			$fxs \leftarrow fxm$\\}
			{$x_i \leftarrow x_m$\\
			$fxi \leftarrow fxm$\\}
			$x_{aux} \leftarrow x_m$\\
			$x_m \leftarrow \frac{x_i + x_s}{2}$\\
			$fxm \leftarrow f(x_m)$\\
			$error \leftarrow |x_m - x_{aux}|$\\
			$contador \leftarrow contador + 1$
		}
		\uIf{$fxm = 0$}
		{$x_m$ es raíz}
		\uElseIf{$error < tolerancia$}
		{$x_m$ es aproximación a una raíz con una tolerancia = $tolerancia$}
		\Else
		{Fracasó en $niter$ iteraciones}
	}
	\Else
	{El intervalo es inadecuado}
\end{algorithm}
\subsection{Método de la regla falsa}
\begin{algorithm}[H]
	\caption{Método de la Regla Falsa}
	\SetAlgoLined
	Leer $x_i$, $x_s$, $tolerancia$, $niter$ \\
	$fxi \leftarrow f(x_i)$ \\
	$fxs \leftarrow f(x_s)$ \\
	\uIf{$fxi = 0$}
	{$x_i$ es raíz}
	\uElseIf{$fxs = 0$}
	{$x_s$ es raíz}
	\uElseIf{$fxi * fxs < 0$}
	{
		$x_m \leftarrow x_i - \frac{f(x_i)*(x_s - x_i)}{f(x_s) - f(x_i)}$\\
		$fxm = f(x_m)$\\
		$contador \leftarrow 1$\\
		$error \leftarrow tolerancia + 1$ \\
		\While{$error > tolerancia$ \textbf{y} $fxm \neq 0$ \textbf{y} $contador < niter$}
		{
			\eIf{$fxi * fxm < 0$}
			{$x_s \leftarrow x_m$\\
				$fxs \leftarrow fxm$\\}
			{$x_i \leftarrow x_m$\\
				$fxi \leftarrow fxm$\\}
			$x_{aux} \leftarrow x_m$\\
			$x_m \leftarrow x_i - \frac{f(x_i)*(x_s - x_i)}{f(x_s) - f(x_i)}$\\
			$fxm \leftarrow f(x_m)$\\
			$error \leftarrow |x_m - x_{aux}|$\\
			$contador \leftarrow contador + 1$
		}
		\uIf{$fxm = 0$}
		{$x_m$ es raíz}
		\uElseIf{$error < tolerancia$}
		{$x_m$ es aproximación a una raíz con una tolerancia = $tolerancia$}
		\Else
		{Fracasó en $niter$ iteraciones}
	}
	\Else
	{El intervalo es inadecuado}
\end{algorithm}
\subsection{Método de Punto Fijo}
\begin{algorithm}[H]
	\caption{Método de Punto Fijo}
	\SetAlgoLined
	Leer $tolerancia$, $x_a$, $niter$\\
	$fx \leftarrow f(x_a)$\\
	$contador \leftarrow 0$ \\
	$error \leftarrow tolerancia + 1$\\
	\While{$fx \neq 0$ \textbf{y} $error > tolerancia$ \textbf{y} $contador < niter$}
	{
		$x_n \leftarrow g(x_a)$\\
		$fx \leftarrow f(x_n)$\\
		$error \leftarrow |x_n - x_a|$\\
		$x_a \leftarrow x_n$\\
		$contador \leftarrow contador + 1$\\
	}
	\uIf{$fx = 0$}
	{$x_a$ es raíz}
	\uElseIf{$error < tolerancia$}
	{$x_a$ es aproximación con una tolerancia = $tolerancia$}
	\Else
	{El método fracasó en $niter$ iteraciones}
\end{algorithm}
\subsection{Método de Newton}
\begin{algorithm}[H]
	\caption{Método de Newton}
	\SetAlgoLined
	Leer $tol$, $x_0$, $niter$\\
	$fx \leftarrow f(x_0)$\\
	$dfx = f'(x_0)$\\
	$contador \leftarrow 0$ \\
	$error \leftarrow tol + 1$\\
	\While{$fx \neq 0$ \textbf{y} $dfx \neq 0$ \textbf{y} $error > tol$ \textbf{y} $contador < niter$}
	{
		$x_1 \leftarrow x_0 - \frac{fx}{dfx}$\\
		$fx \leftarrow f(x_1)$\\
		$dfx \leftarrow f'(x_1)$\\
		$error \leftarrow |x_1 - x_0|$\\
		$x_0 \leftarrow x_1$\\
		$contador \leftarrow contador + 1$\\
	}
	\uIf{$fx = 0$}
	{$x_a$ es raíz}
	\uElseIf{$error < tol$}
	{$x_a$ es aproximación con una tolerancia = $tolerancia$}
	\uElseIf{$dfx = 0$}
	{$x_1$ es una posible raíz múltiple}
	\Else
	{El método fracasó en $niter$ iteraciones}	
\end{algorithm}
\end{document}