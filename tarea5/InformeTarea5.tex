\documentclass[12pt]{article}

\title{Informe Tarea 5}

\author {Ronald Cardona
\and Anderson Grajales
\and Sebastian Valencia
\and Julian Sanchez}


% Packages
\usepackage[spanish]{babel}
\selectlanguage{spanish} 
\usepackage[utf8]{inputenc}
\usepackage[spanish, onelanguage]{algorithm2e} %for psuedo code
\usepackage{amsthm}
\usepackage{amsmath}

\newtheorem{definition}{Definición}[section]
\newtheorem{theorem}{Teorema}[section]



\begin{document}
	\maketitle
	\section{Solución numérica de sistemas de ecuaciones lineales}
	Muchos problemas del mundo real se formulan como sistemas de ecuaciones de $n$ variables y $m$ incógnitas que bajo condiciones ideales($n$ y $m$ no son valores muy grandes), se pueden resolver de manera analítica. Sin embargo, cuando $n$ y $m$ tienden a ser valores muy grandes la solución analítica a estos problemas es muy difícil de calcular ya que requiere de mucho tiempo y claramente no es la forma más eficiente hacerlo. Debido a esto, desde el campo del \textit{Análisis Numérico} se plantean diversas formas computacionales, ya sean algoritmos u otras técnicas que nos permitan resolver estos sistemas rápidamente, teniendo en cuenta que hay una \textit{\textbf{propagación de error}} en cada cálculo dependiendo de la capacidad de la computadora donde se ejecuten estos.
	En este documento se presentan algunos algoritmos numéricos que son de gran ayuda a la hora de resolver sistemas de ecuaciones lineales.
	\subsection{Determinantes}
	\begin{definition}
		Sea $A = [a_{ij}]$ una matriz de tamaño $n \times n$. El cofactor $C_{ij}$ de $a_{ij}$ se define como $(-1)^{i+j}\det {M_{ij}}$, donde $M_{ij}$ es la matriz de tamaño $(n - 1) \times (n - 1)$, que se obtiene al eliminar la fila $i$ y la columna $j$ de la matriz.\cite{algebralineal}
	\end{definition}
	\begin{theorem}
		Sea $A=[a_{ij}]$ una matriz de tamaño $n \times n$. \cite{algebralineal}
		\begin{itemize}
			\item Para cada $1\leq i \leq n$ se cumple que:
			$\det {A} = a_{i1}C_{i1} + a_{i2}C_{i2} + ... + a_{in}C_{in}$
			\item Para cada $1\leq j \leq n$ se cumple que:
			$\det {A} = a_{1j}C_{1j} + a_{2j}C_{2j} + ... + a_{nj}C_{nj}$
		\end{itemize}
	\end{theorem}
	De acuerdo al teorema 1.1 se puede definir una ecuación de recurrencia para encontrar el determinante de una matriz $A = [a_{ij}]$ de la siguiente manera:
	
	\begin{equation}
	det(A, n)_{1\leq i \leq n}=\begin{cases}
	a_{11}, & \text{if $n = 1$}.\\
	(-1)^{i+1} \times a_{1i} \times det(A', n - 1) , & \text{if $n > 1$}.
	\end{cases}
	\end{equation}
	Donde $A'$ es la matriz que se obtiene al eliminar la columna $i$ y la fila $n$ de $A$. De esta manera, para una matriz $B$ de $n \times n$, la solución se entrega de la forma:
	$det(B, n)$.
	
	\subsection{Multiplicación de matrices}
	\begin{definition}
		Dadas las matrices $A \in M_{m\times n}$ y $B \in M_{n\times p}$, entonces el producto de $A$ con $B$, denotado $AB$, es una matriz $C \in M_{m\times p}$, dada por: \cite{algebralineal}
		\[
			c_{ij} = a_{i1}b_{1j} + a_{i2}b_{2j} + a_{i3}b_{3j} + \dots + a_{in}b_{nj} = \sum_{k = 1}^{n}{a_{ik}b_{kj}}		
		\]
		con $i = 1, ..., m$ y $j = 1, ..., p$
	\end{definition}
	\subsection{Escalonamiento de matrices}
	\begin{definition}
		Sea $A = [a_{ij}]$ una matriz de $n \times n$. Decimos que $A$ está escalonada si $\forall{i,j}, 1\leq i \leq j \leq n$, $a_{ij} = 0$.
	\end{definition}
	\begin{algorithm}[H]
		\caption{Algoritmo para escalonar matrices}
		Leer $A$, $b$ \\
		\uIf{$A \not \in \Re^{n\times n}$ \textbf{ó} $b \not\in \Re^{n}$}
		{$A$ debe ser cuadrada y $b$ debe ser un arreglo de $n$ posiciones}
		\uElseIf{$\det{A} = 0$}
		{$A$ debe ser invertible}
		\Else
		{
			\For{$k = 1$ \KwTo $n-1$}
			{
				\If{$A_{kk} = 0$}
				{
					$j \leftarrow k + 1$\\
					\While{$j < n$ \textbf{y} $A_{jk} = 0$}
					{
						$j \leftarrow j + 1$
					}
					\If{$j < n$}
					{
						\For{$l = k$ \KwTo $n$}
						{
							$A_{kl} \leftarrow A_{kl} + A_{jl}$
						}
						$b_k \leftarrow b_k + b_j$
					}
				}
				\For{$i = k + 1$ \KwTo $n$}
				{
					\If{$A_{ki} \not = 0$}
					{
						$m \leftarrow \frac{A_{ik}}{A_{kk}}$\\
						\For{$l = k$ \KwTo $n$}
						{
							$A_{il} \leftarrow A_{il} - m\times A_{kl}$
						}
						$b_i \leftarrow b_i - m \times b_k$
					}
				}
			}
		}
		La solución $(A, b)$
	\end{algorithm}
	\begin{thebibliography}{9}
		\bibitem{algebralineal}
		Orlando García Jaimes, Jairo A. Villegas Gutiérrez, Jorge Iván. \textit{Álgebra Lineal}. Editorial EAFIT, Medellín 2012. 
	\end{thebibliography}
\end{document}